% Learning Basics - learningBasics.tex
% Created by Minhduc Cao for ICS 6B (Gassko)
% ~ A simplified guide to learning basic LaTeX for writing assignments 

% ================= READ ME =================
% 1) I just started learning LaTeX so there might be a few errors (:
% 2) This document shows you a select few basics to get you started
% 3) Use myTemplate.tex to get started on writing your assignments
% 4) Use symbols.tex to copy commonly used symbols in this course

\documentclass[letterpaper]{article}    % Sets size of document to US letter paper
\usepackage[margin=1in]{geometry}       % Sets margin to 1 inch to give you more space for work
\usepackage{times}                      % Font
\usepackage{amssymb}                    % Gives you extra logic symbols 
%\hbadness=99999                        % Delete the % at the beginning to removes typeset error notifications (usually not a big deal)
\begin{document}                        % Where the document starts

% >>>>> TITLE AND KEY INFO <<<<<
\title{Title of Assignment}             % Title should be in the curly brackets
\author{Your Name Here\\                % You can use \\ to move to the next line for text, it can be placed before or after text depending on what you want
        Other Details Here\\
        ICS 6B - Lecture A/B}
\date{\today}                           % You can manually write in a date as well
\maketitle                              % You need this otherwise the title won't show up

% >>>>> EXAMPLE QUESTION LAYOUT <<<<<
\section*{1.1.1}                        % The asterisk removes the automatic numbering that LaTeX does
\subsection*{a.}                        % Add subsections to space out the questions in each section
Insert work here
\subsection*{b.}
$(x \land y) \to z$                     % Example logical statement; NOTE: expressions that use logical operators MUST have a DOLLAR SIGN at the beginning and end
\pagebreak                              % HIGHLY RECOMMENDED to create a new page for the following question
                                        % Not doing this will make the next question appear right under your last one (harder for graders)

% >>>>> LESS IMPORTANT INFO <<<<<
\begin{center}                          % Centers a block of info
\section{Less Important info}           % Leaving out the asterisk (*) gives auto-numbers the section
% Comment text                          % Having the % sign before a line makes it a comment which is ignored by LaTeX
non-bold \textbf{bolded text}           % Bolded text
\\\textit{italicized text}              % Italicized text
\subsubsection{Subsubsection Header}    % Auto-numbered subsubsection header
\end{center}                            % Closes block of info to be centered

\end{document}                          % Where the document ends
